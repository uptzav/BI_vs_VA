\section{Perspectivas de expertos} 

Encuestamos a siete expertos líderes en todo el espectro de inteligencia empresarial para comprender mejor la distinción entre estos dos términos de referencia.
\vspace{6mm} %5mm vertical space

\textbf{Mantener vs. revolucionar}
Se necesita Business Intelligence para administrar el negocio, mientras que Business Analytics es necesario para cambiar el negocio.
BI se enfoca en crear eficiencia operativa a través del acceso a datos en tiempo real que permiten a las personas realizar de manera más efectiva sus funciones de trabajo.\\ BI también incluye el análisis de datos históricos de múltiples fuentes que permiten una toma de decisiones informada, así como la identificación y resolución de problemas.\\Business Analytics se relaciona con la exploración de datos históricos de muchos sistemas fuente a través del análisis estadístico, análisis cuantitativo, minería de datos, modelado predictivo y otras tecnologías y técnicas para identificar tendencias y comprender la información que puede impulsar el cambio comercial y respaldar prácticas comerciales exitosas sostenidas.
\begin{flushleft}
Pat Roche\\  
Vicepresidente de Ingeniería, Noetix Products\\  
Magnitude Software
\end{flushleft}

\textbf{Comprender el pasado versus el futuror}
Para mí, la diferencia entre Business Intelligence es mirar por el espejo retrovisor y usar datos históricos desde hace un minuto hasta hace muchos años. Business Analytics está mirando en frente de usted para ver qué va a pasar. Esto te ayudará a anticipar lo que viene, mientras que BI te dirá lo que sucedió. Esta es una distinción muy importante ya que ambos le proporcionarán diferentes puntos de vista, no menos. La BI es importante para mejorar su toma de decisiones sobre la base de resultados pasados, mientras que los análisis de negocios le ayudarán a avanzar y comprender lo que podría suceder.
\begin{flushleft}
Mark van Rijmenam\\  
CEO / Fundador\\  
BigData-Startups
\end{flushleft}

\textbf{Informes y análisis}
La inteligencia empresarial tradicional (BI) se ha centrado principalmente en la elaboración de informes. En este enfoque de BI, algunas personas crean informes con muchos formatos (por lo general, los desarrolladores de informes) y se distribuyen a todo un departamento u organización. Más recientemente, la tendencia en análisis ha sido brindar a las personas que tienen preguntas sobre sus datos las herramientas para obtener sus propias respuestas. Ahora se trata de dejar que las personas de negocios se conviertan en analistas.\\ Esto a menudo se conoce como "análisis de autoservicio", y en este enfoque no se trata solo de generar informes, sino de permitir que las personas entren en el flujo del análisis, exploren sus datos y formulen sus propias preguntas. Esto ha cambiado por completo la forma en que muchas empresas se acercan a la inteligencia de negocios.
\begin{flushleft}
Francois Ajenstat \\  
Tableau
\end{flushleft}